% #####################################
% THESIS TEMPLATE (HBRS BACHELOR & MASTER)
% ----> content/99_Erklaerung.tex
% Author: Ilka Salomon
% E-Mail: ilka.salomon@gmx.de
% #####################################


\documentclass[12pt,a4paper,twoside, xcolor=dvipsnames, openright]{book}

%\usepackage[doublespacing]{setspace}
% INCLUDES ------------------------------------------------------
\usepackage{glossaries}
\usepackage{fancyhdr}
\usepackage{xcolor,colortbl}
\usepackage{pgffor}
\usepackage{blindtext}
\usepackage{ifthen,changepage}
\usepackage[english]{babel}
\usepackage{mathptmx}
\usepackage[toc,page,titletoc]{appendix}
\usepackage[scaled]{helvet}
\usepackage[utf8]{inputenc}
\usepackage[T1]{fontenc}
\usepackage{titlesec}
\usepackage{blindtext}
\usepackage[includeheadfoot, a4paper,top=2cm,bottom=2cm,left=4cm,right=2cm]{geometry}
%\usepackage {extsizes}
\usepackage{graphicx}
\usepackage {ifthen}
\usepackage{tocloft} 
\usepackage{tabularx}
\usepackage{forest}
\usepackage{booktabs}
\usepackage{url}
\usepackage[hidelinks]{hyperref}
\usepackage{etoolbox}
\usepackage{glossary-mcols}
\usepackage{float}
%\usepackage{multibbl}
%\makeatother
\usepackage[square,sort,comma,numbers]{natbib}
\usepackage{multibib}
\newcites{ltex}{Online Sources}
\def\@mb@citenamelist{cite,citep,citet,citealp,citealt,citepltex,citetltex}
\usepackage{amsfonts}
\usepackage{mathrsfs}
\usepackage{svg}
\usepackage{pifont}
\usepackage{xcolor}
\usepackage{amsthm}
\usepackage{framed}
\usepackage{scrextend}
\usepackage[most]{tcolorbox}
\usetikzlibrary{shadows}
\usetikzlibrary[shadows]
\usepackage{trfsigns}
\usepackage{bigints}
\usepackage{relsize}
\usepackage{cancel}
\usepackage{listings}
\usepackage{tcolorbox} 
\usepackage{pgfgantt}
\usepackage[margin=10pt,font=small,labelfont=bf,
labelsep=endash]{caption}
\usepackage{subcaption}
\usepackage{fontawesome}
\usepackage{adjustbox}
\usepackage{forest}
\usepackage{multirow}
\usepackage{xspace}
\usepackage{electrum}
\usepackage{enumitem}
\usepackage{arydshln}
\usepackage{nameref}
\appto\UrlBreaks{\do\-}
\usepackage[onehalfspacing]{setspace}
%\usepackage{libertine}
% ###############################################
% COLORS
% ###############################################

% HBRS Color Scheme --------
\definecolor{hbrs_scheme1}{HTML}{00ACFB}
%\setbeamercolor{hbrs_scheme1}{fg=hbrs_scheme1,bg=white}
\definecolor{hbrs_scheme2}{HTML}{0099E0}
\definecolor{hbrs_scheme3}{HTML}{007DB8}
\definecolor{hbrs_scheme4}{HTML}{00628F}
\definecolor{hbrs_scheme5}{HTML}{004666}
\definecolor{todo_scheme}{HTML}{FF8800}
\definecolor{hbrs_scheme6}{HTML}{b8d0db}
% -------------------------
% MATLAB Color Scheme --------
\definecolor{matlab_blue}{HTML}{0099E0}
% DRAW IO Color Scheme --------
\definecolor{drawio_orange}{HTML}{D79B00}
\definecolor{drawio_red}{HTML}{B85450}
\definecolor{drawio_green}{HTML}{82B366}
\definecolor{drawio_blue}{HTML}{6C8EBF}


\definecolor{folderbg}{RGB}{124,166,198}
\definecolor{folderborder}{RGB}{110,144,169}

\def\Size{4pt}
\tikzset{
  folder/.pic={
    \filldraw[draw=folderborder,top color=folderbg!50,bottom color=folderbg]
      (-1.05*\Size,0.2\Size+5pt) rectangle ++(.75*\Size,-0.2\Size-5pt);  
    \filldraw[draw=folderborder,top color=folderbg!50,bottom color=folderbg]
      (-1.15*\Size,-\Size) rectangle (1.15*\Size,\Size);
  }
}


\newcommand*\circled[1]{\tikz[baseline=(char.base)]{
            \node[shape=circle,draw,inner sep=2pt] (char) {#1};}}

\newcommand\nd{\textsuperscript{nd}\xspace}
\newcommand\st{\textsuperscript{st}\xspace}
\newcommand\rd{\textsuperscript{rd}\xspace}


% SETTINGS ------------------------------------------------------

	% Language ____________________________

	\newif\ifen
	\newif\ifde
	\newcommand{\en}[1]{\ifen#1\fi}
	\newcommand{\de}[1]{\ifde#1\fi}
	% setting:
	\entrue

	% Date and Time ____________________________
	
	\usepackage{datetime}
	\newdateformat{englishDate}{%
	\the\day
	\ifthenelse {\equal{\the\day}{1}}{st}{\ifthenelse {\equal{\the\day}{2}}{nd}{\ifthenelse {\equal{\the\day}{3}}{rd}{th}}}
	\monthname[\THEMONTH], \THEYEAR}


	% Font Basics ____________________________
	\renewcommand\familydefault{\sfdefault}
	\setlength{\parindent}{0in}

	% Chapters ______________________________
	\patchcmd{\chapter}{\thispagestyle{plain}}{\thispagestyle{fancy}}{}{}
	\titleformat{\chapter}
	  {\normalfont\fontsize{16pt}{16pt}\selectfont\bfseries\color{black}}
	  {\thechapter}{16pt}{\fontsize{16pt}{16pt}\selectfont}
	\titlespacing*{\chapter}{0pt}{12pt}{12pt}

	% Sections ______________________________
	\titleformat{\section}
	  {\normalfont\fontsize{13pt}{13pt}\selectfont\bfseries\color{black}}
	  {\thesection}{13pt}{\fontsize{13pt}{13pt}\selectfont}
	\titlespacing*{\section}{0pt}{12pt}{6pt}
	
	% Subsections ______________________________
	\titleformat{\subsection}
	  {\normalfont\fontsize{12pt}{12pt}\selectfont\bfseries\color{black}}
	  {\thesubsection}{12pt}{\fontsize{12pt}{12pt}\selectfont}
	\titlespacing*{\subsection}{0pt}{6pt}{6pt}
	
	 %Subsubsections ______________________________
	%\titleformat*{\subsubsection}[runin]{\color{hbrs_scheme3}\bfseries}{}{}{\\}[ | \hspace*{2cm}]
	\titleformat{\subsubsection}
	{\normalfont\fontsize{12pt}{12pt}\selectfont\bfseries\color{hbrs_scheme4}}
	  {\thesubsection}{12pt}{\fontsize{12pt}{12pt}\selectfont}
	\titlespacing*{\subsubsection}{0pt}{6pt}{4pt}
	
	% Headers/Footers _______________________

	\renewcommand{\chaptermark}[1]{\markboth{#1}{}}
	\fancypagestyle{normalpage} {
		\fancyfoot[E]{}                            % Delete current footer settings
		\fancyhead[E]{}
		\fancyfoot[O]{}                            % Delete current footer settings
		\fancyhead[O]{}
		\fancyhf{}
		\setlength{\headheight}{1cm}
		\setlength{\headsep}{25pt}
		%\setlength{\topmargin}{-1.5cm}
		%\setlength{\voffset}{3.5cm}
		\renewcommand{\footrulewidth}{0.5pt}
		\fancyfoot[RO,LE]{\bfseries\thepage}    % Page number (boldface) in left on even
	
		%\fancyhead[LO]{\includegraphics [width=6cm]{logo_hbrs}}      % Logo in the left on odd pages
		%\fancyhead[LO]{\includegraphics [width=3cm]{logo_ford}}      % Logo in the left on odd pages
		\fancyhead[RO]{\includegraphics [width=6cm]{logo_hbrs} \hspace {0.5cm} \includegraphics [width=3cm]{logo_ford}}      % Logo in the left on odd pages
		%\fancyhead[LO]{\includegraphics [width=3cm]{logo_ford} \hspace {0.5cm} \includegraphics [width=6cm]{logo_hbrs} }      % Logo in the left on odd pages
		\fancyhead[LE]{\bfseries\nouppercase{\leftmark} }                 % chaptername in the right on even pages
		%\fancyhead[C] {\includegraphics [width=2cm]{logo_ford}}
	}
	
	%\tableofcontents{\thispagestyle{normalpage}}
	
	% Figures  _________________________________
		\renewcommand{\thefigure}{\thesection-\arabic{figure}}
		\addto\captionsgerman{%
			\renewcommand{\figurename}%
			{\de{Abbildung}\en{Figure}}%
		}
		\numberwithin{figure}{section}

	% Tables  _________________________________
		\renewcommand{\thetable}{\thesection-\arabic{table}}
		\renewcommand*{\arraystretch}{1.2}

		\addto\captionsgerman{%
		\renewcommand{\tablename}%
		{\de{Tabelle}\en{Table}}%
		}

	% Listing  _________________________________
	\AtBeginDocument{\renewcommand*{\thelstlisting}{\thesection-\arabic{lstlisting}}}
	\renewcommand{\lstlistingname}{\vspace{4mm}\bfseries\itshape{Listing}}
	\renewcommand{\lstlistlistingname}{List of Listings}% List of Listings -> List of Algorithms
	\setlength{\cftfignumwidth}{4 em}
	\setlength{\cfttabnumwidth}{4 em}
	\captionsetup[lstlisting]{format=hang, font=small, labelfont=bf, indention=-2.5cm}

	\lstdefinelanguage{seesharp}{
		basicstyle=\scriptsize,
		frame=tb,
		emph=[1]%
    {%
    SimpleIoc, new, Binding, Environment, NewCloudDiagnosticsChunkAvailable, NewCloudCommunicationChunkAvailable,
	DataExplorerConnectionContext, CvbopConnectionContext, Fields, Properties, Methods, \\\\
    },
    emphstyle=[1]{\bfseries\color{drawio_green}},
    %
    emph=[2]% 
    {% 
	MeasurementManager, CloudCommuncationService, CloudDiagnosticService, Token, Uri, HttpClient, ValidationRule, CloudProvisioningStatusEntry, CloudProvisioningAuthorizationStatus,
	ICloudDiagnosticBase, NewCloudDiagnosticsChunkAvailableArgs, AuthenticationHeaderValue, Result,
	StaticResource, CloudEnrollmentStatus, CloudEnrollmentStatusMapper, EnrollmentDetailsHolder, IEnumerable, List,
    public, private, string, bool, RelayCommand, void, Button, return, null, ViewModelClass, ModelClass, Dictionary,
	CvbopApiV2Repository,DataExplorerApiV2Repository, IDataExplorerRepository, ICvbopRepository, IRestClient
    },
	emphstyle=[2]{\bfseries\color{drawio_blue}},
	%
	emph=[3]% 
	{% 
	Register, GetInstance, GetCloudDiagnosticService, GetCloudCommunicationService, ValidateProvisioningStatus,
	PrepareAndInvokeMessageChunks, UpdateDiagnostics, MethodName, Create,
	GetCloudEnrollmentStatus, FindEnrollmentDetails, Map, MapBack, Get, Put, Converter,
	SetRestClient, UpdateDiagnostics
	},
	emphstyle=[3]{\bfseries\color{drawio_orange}},
	%
	emph=[4]% 
	{% 
	get, set, _cvbopRepository, _dataExplorerRepository, _restClient, _cloudDiagnosticService
	},
	emphstyle=[4]{\bfseries\color{drawio_red}},
	}

	% Table of Contents  _______________________
	\renewcommand\cfttoctitlefont{\normalfont\fontsize{16pt}{16pt}\selectfont\bfseries}
	\setlength{\cftpartindent}{0em}
	\renewcommand{\cftchapdotsep}{\cftdotsep}
	\renewcommand{\cftsecdotsep}{\cftdotsep}
	\renewcommand{\cftsubsecdotsep}{\cftdotsep}
	\renewcommand{\cftsubsubsecdotsep}{\cftdotsep}
	\setlength{\cftbeforetoctitleskip}{12pt}
	\setlength{\cftbeforeloftitleskip}{12pt}
	\setlength{\cftbeforelottitleskip}{12pt}
	%\renewcommand\l@lstlisting[2]{\@dottedtocline{1}{1.5em}{4em}{~#1}{#2}}
	
	\renewcommand{\cftpartpagefont}{\cftchappagefont} % Heading 1 / Roman Nr
	\renewcommand{\cftchappagefont}{\cftsecpagefont} % Heading 1 / Roman Nr
	%\setlength{\cftbeforechapskip} {6pt} % Heading 1 / Arabic Nr
	\setlength{\cftbeforechapskip} {8pt} % Heading 1 / Arabic Nr
	\setlength{\cftbeforesecskip} {6pt} % Heading 2
	\setlength{\cftbeforesubsecskip} {6pt} % Heading 3
	\setlength{\cftbeforesubsubsecskip} {6pt} % Heading 4

	% List of Figures
	
	\renewcommand\cftloftitlefont{\normalfont\fontsize{16pt}{16pt}\selectfont\bfseries}
	\renewcommand\cftfigfont{\bfseries}
	%\renewcommand\cftfigpagefont{\Large}
	\renewcommand{\cftfigdotsep}{\cftdotsep}
	
	% List of Tables
	
	\renewcommand\cftlottitlefont{\normalfont\fontsize{16pt}{16pt}\selectfont\bfseries}
	\renewcommand\cfttabfont{\bfseries}
	%\renewcommand\cfttabpagefont{\Large}
	\renewcommand{\cfttabdotsep}{\cftdotsep}

	% List of Listings
	\makeatletter
	\renewcommand*{\l@lstlisting}[2]{\@dottedtocline{1}{1.5em}{3.5em}{\bfseries#1}{#2}}
	\let\my@chapter\@chapter
	\renewcommand*{\@chapter}{%
	  \addtocontents{lol}{\protect\addvspace{10pt}}%
	  \my@chapter}
	  \addtocontents{lol}{\protect\addvspace{30pt}}
	\makeatother


	% Glossary Style General   _______________________
	
	%\makenoidxglossaries
	\makeglossaries
	\setacronymstyle{long-short}
	\setglossarystyle{alttreegroup} % 1 Column
	%\setglossarystyle{mcolalttreegroup} % 2 Columns
	%\renewcommand{\glsnamefont}[1]{\normalfont\fontsize{12pt}{12pt}\selectfont{\bfseries #1}}
	%\renewcommand{\glsnamefont}[1]{\normalfont\fontsize{12pt}{12pt}\selectfont{#1}}

	% Setup Glossary Divider   _______________________
	\renewcommand*{\glsgroupheading}[1]{\par
	\def\@gls@prevlevel{-1}%
	\hangindent0pt\relax
	\parindent0pt\relax
	\glstreegroupheaderfmt{\normalfont\bfseries\fontsize{14pt}{14pt}\selectfont\glsgetgrouptitle{#1} \rule{\textwidth}{0.01pt}}\par\indexspace} % Line as Divider
	% \glstreegroupheaderfmt{\normalfont\bfseries\fontsize{14pt}{14pt}\selectfont\glsgetgrouptitle{#1} \rule{120pt}{0.4pt}}\par\indexspace} % Line as Divider, short
	%\glstreegroupheaderfmt{\glsgetgrouptitle{#1} \vspace{5pt} \dotfill}\par\indexspace} % Dots as Divider
	% Column Seperation ------------
	\setlength{\columnsep}{1cm}
	\setlength{\columnseprule}{1pt}
	%\newglossary[slg]{symbols}{sym}{sbl}{Symbols}
	%\loadglsentries{gls_symbols}
	%\newglossary[alg]{acronyms}{acr}{acn}{Abkürzungsverzeichnis}
	\loadglsentries{gls_acronyms}
	\glssetwidest{sanhhhhkkkhhheck}
	\glsenablehyper
	\makeglossaries
% MACROS ------------------------------------------------------------

	% Citations _________________________________

		% Block Citation
		% \blockCitation {content}
			\newcommand {\BlockCitation} [1]
			{\par \begingroup \leftskip 1cm \begin{tabular}
			{| p{12cm}} {#1} \end {tabular} \\ \par \endgroup}

		% Text Indent 1cm
		% \indentText {content}
			\newcommand {\indentText} [1]
			{\par \begingroup \leftskip 1cm \begin{tabular}
			{p{12cm}} {#1} \end {tabular} \leftskip 0cm \par \endgroup}

	% FIGURES _________________________________

		% Add Figure
		% \addFigure [opt. width multiplier / standard = 0.8 * \textwidth]{File Name = Lable Name}{Caption}{Position}
		\newcommand {\addFigure} [5] [0.8]
		{\begin {figure} [#5] \center
		\includegraphics [width=#1\textwidth] {#2} 
		\bfseries\scriptsize\itshape{\caption[\normalfont#4]{\mdseries\footnotesize\itshape{#3}}\label{fig:#2}}
		\end {figure}}
		
		% Add Figure
		% \addRotatedFigure [opt. width multiplier / standard = 0.8 * \textwidth]{File Name = Lable Name}{Caption}{Position}
		\newcommand {\addRotatedFigure} [5] [0.8]
		{\begin {figure} [#5] \center
		\includegraphics [width=#1\textwidth, angle =90] {#2} 
		\bfseries\scriptsize\itshape{\caption[\normalfont#4]{\mdseries\footnotesize\itshape{#3}}\label{fig:#2}}
		\end {figure}}
		
		% Add SVG Figure
		% \addFigureSVG [opt. width multiplier / standard = 0.8 * \textwidth]{File Name = Lable Name}{Caption}{Position}
		\newcommand {\addFigureSVG} [5] [0.8]
		{\begin {figure} [#5] \center
		\includesvg [width=#1\textwidth] {#2} 
		\bfseries\scriptsize\itshape{\caption[\normalfont#4]{\mdseries\footnotesize\itshape{#3}}\label {fig:#2}}
		\end {figure}}
		
		% Add SVG Figure
		% \addRotatedFigureSVG [opt. width multiplier / standard = 0.8 * \textwidth]{File Name = Lable Name}{Caption}{Position}
		\newcommand {\addRotatedFigureSVG} [5] [0.8]
		{\begin {figure} [#5] \center
		\includesvg [width=#1\textwidth, angle =90] {#2} 
		\bfseries\scriptsize\itshape{\caption[\normalfont#4]{\mdseries\footnotesize\itshape{#3}}\label {fig:#2}}
		\end {figure}}
		
		% Add Wrap Figure
		% \addWrapFigure  {width wrap thing multiplier: width = x * \textwidth} {width figure multiplier: width = x * \textwidth}{File Name = Lable Name} {Caption}
		
		\usepackage{wrapfig}
		\newcommand {\addWrapFigure} [5]
		{\par\begin{wrapfigure}{L}{#1\textwidth}
		\centering
		\includegraphics[width=#2\textwidth]{#3}
		\bfseries\scriptsize\itshape{\caption[\normalfont#4] {\mdseries\footnotesize\itshape{#5}}
		\label{fig:#3}}
		\end{wrapfigure}\par}
		
		\newcommand {\addWrapFigureSVG} [5]
		{\par\begin{wrapfigure}{L}{#1\textwidth}
		\centering
		\includesvg [width=#2\textwidth]{#3}
		\bfseries\tiny\itshape{\caption[\normalfont#4] {\mdseries\footnotesize\itshape{#5}}}
		\label{fig:#3}\end{wrapfigure}\par}
		

		% Add Figure
		% \addFigureTrim [opt. width multiplier / standard = 0.8 * \textwidth]{File Name = Lable Name}{Caption}{Position}
		\newcommand {\addFigureTrim} [6] [0.8]
		{\begin {figure} [#5] \center
		\includegraphics [width=#1\textwidth,trim= #6, clip] {#2} 
		\bfseries\scriptsize\itshape{\caption[\normalfont#4]{\mdseries\footnotesize\itshape{#3}}\label {fig:#2}}
		\end {figure}}
		% AddTwoFigures
		% \addTwoFigures {File Name 1 = Lable Name 1}{Caption 1} {File Name 2 = Lable Name 2}{Caption 1}
		
		\newcommand {\addTwoFigures} [8] 		
		{
			\begin{figure} [#7]
			\centering
			\begin{subfigure}{.5\textwidth}
			  \centering
			  \includegraphics[width=#6\linewidth]{#1}
			  \bfseries\scriptsize\itshape{\caption[\normalfont#2] {\mdseries\footnotesize\itshape{#2}}\label{fig:#1}}
			  
			\end{subfigure}%
			\begin{subfigure}{.5\textwidth}
			  \centering
			  \includegraphics[width=#6\linewidth]{#3}
			  \bfseries\scriptsize\itshape{\caption[\normalfont#4] {\mdseries\footnotesize\itshape{#4}} \label{fig:#3}}
			 
			\end{subfigure}
			\bfseries\scriptsize\itshape{\caption[\normalfont#8] {\mdseries\footnotesize\itshape{#5}}\label{fig:#1_and_#3}}
			
			\end{figure}
		}

		% Provide a way to declare and renew a command in one command
		\newcommand{\neworrenewcommand}[1]{\providecommand{#1}{}\renewcommand{#1}}

		\newcommand{\addThreeFigures}[9]{
			\neworrenewcommand{\TmpAddThreeFigures}[1]{
				\begin{figure} [#9]
					\centering
					\begin{subfigure}{.3\textwidth}
					\centering
					\includegraphics[width=#8\linewidth]{#1}
					\bfseries\scriptsize\itshape{\caption[\normalfont#2] {\mdseries\footnotesize\itshape{#2}}\label{fig:#1}}
					\end{subfigure}%
					\begin{subfigure}{.3\textwidth}
					\centering
					\includegraphics[width=#8\linewidth]{#3}
					\bfseries\scriptsize\itshape{\caption[\normalfont#4] {\mdseries\footnotesize\itshape{#4}} \label{fig:#3}}
					
					\end{subfigure}%
					\begin{subfigure}{.3\textwidth}
						\centering
						\includegraphics[width=#8\linewidth]{#5}
						\bfseries\scriptsize\itshape{\caption[\normalfont#6] {\mdseries\footnotesize\itshape{#6}} \label{fig:#5}}
					\end{subfigure}
					\bfseries\scriptsize\itshape{\caption[\normalfont##1] {\mdseries\footnotesize\itshape{#7}}\label{fig:#1_and_#3_and#5}}
					
					\end{figure}
			}
			\TmpAddThreeFigures
		}

		\newcommand {\addThreeFiguresSvg} [9] 		
		{
			\neworrenewcommand{\TmpAddThreeFiguresSvg}[1]{
			\begin{figure} [#9]
			\centering
			\begin{subfigure}{.3\textwidth}
			  \centering
			  \includesvg[width=#8\linewidth]{#1}
			  \bfseries\scriptsize\itshape{\caption[\normalfont#2] {\mdseries\footnotesize\itshape{#2}}\label{fig:#1}}
			\end{subfigure}%
			\begin{subfigure}{.3\textwidth}
			  \centering
			  \includesvg[width=#8\linewidth]{#3}
			  \bfseries\scriptsize\itshape{\caption[\normalfont#4] {\mdseries\footnotesize\itshape{#4}} \label{fig:#3}}
			 
			\end{subfigure}%
			\begin{subfigure}{.3\textwidth}
				\centering
				\includesvg[width=#8\linewidth]{#5}
				\bfseries\scriptsize\itshape{\caption[\normalfont#6] {\mdseries\footnotesize\itshape{#6}} \label{fig:#5}}
			  \end{subfigure}
			\bfseries\scriptsize\itshape{\caption[\normalfont##1] {\mdseries\footnotesize\itshape{#7}}\label{fig:#1_and_#3_and#5}}
			\end{figure}
			}
			\TmpAddThreeFiguresSvg
		}

	% TABLES _________________________________

		% Add Table
		% \addTable {caption}{label}{vlinestyle}{position}{content}
		% other useful commands: \multirow{n}{*}{Content} <- * for centering
		% \multicolumn {n} {vlinestyle, e.g. |c|} {Content}
		% \cline {n,m}
		\newcommand {\addTable} [6]
		{ \begin {table} [#5]  \bfseries\scriptsize\itshape{\caption[\normalfont#4]{\mdseries\footnotesize\itshape#1 \vspace{-0.3cm}\label {tab:#2} }} 
		\normalfont\fontsize{12pt}{12pt}\selectfont\center \begin {tabular} {#3} \toprule #6 \\ \bottomrule \end {tabular}
		 \end {table}}

		 \newcommand {\addTableRotated} [6]
		 { \begin {table} [#5]  \bfseries\scriptsize\itshape{\caption[\normalfont#4]{\mdseries\footnotesize\itshape#1 \vspace{0cm}\label {tab:#2} }} 
		 \normalfont\fontsize{12pt}{12pt}\selectfont\center 
		 \begin{adjustbox}{angle=90}
		 \begin {tabular} {#3} \toprule #6 \\ \bottomrule \end {tabular}
		 \end{adjustbox}
		  \end {table}}

		\newcolumntype{C}[1]{>{\centering\arraybackslash}m{#1}}

		\newcommand {\doubleRow} [1]
		{\multirow{2}{*}{#1}}

		\newcommand {\doubbleRow} [1]
		{\multirow{2}{p{4cm}}{#1}}

	% ITEMIZE _________________________________

		\newcommand {\addList} [1]
		{\vspace{-0.5cm}\begin{singlespace}\begin{itemize} #1
		\end{itemize}\end{singlespace}}
		\newcommand {\addListNoSep} [1]
		{\begin{itemize}[noitemsep,topsep=0.3em] #1
		\end{itemize}}
		\newcommand {\itm} [1]
		{\item[\faStop]{#1}}
		\newcommand {\iitm}[1]
		{\addList{\itm #1}}

		\newcommand {\ilkaBullet} {
		\item [\color{hbrs_scheme3} \ding{110}]}
		
		
		\newcommand {\ilkaBulletWithDescription} [4] {
		
		\ilkaBullet {
			\begin {tabular}
				{ m{#3} !{\color{hbrs_scheme3}\vrule width 1.5px\hspace{-2px}} !{\color{hbrs_scheme4}\vrule width 1px\hspace{-2px}} !{\color{hbrs_scheme5}\vrule width 0.5px} m{#4} }
				 \textbf{#1} & #2
			\end{tabular}}
		}
		
		\newcommand {\twoColumnList} [2] {
		\vspace{-0.5cm}\\ \vspace{-0.3cm}
		\begin {tabular} {p{6cm} p{6cm}}\begin {itemize}#1\end{itemize}& \begin {itemize}#2\end{itemize}
		\end {tabular}
		}
		
	% QUOTE _________________________________

		\newtcolorbox{quoteBox}[1]{colback=hbrs_scheme1!5!white,colframe=hbrs_scheme4!75!white, leftrule=3mm,drop shadow southeast}

		\newcommand{\AddQuote}[2] {
			\begin{addmargin}[0.5cm]{0.5cm}
				\begin{quoteBox}
					\textit{``#1''{} #2}
				\end{quoteBox}
			\end{addmargin}}
%\small\color{hbrs_scheme5}\url{" * url * "}\normalsize\color{black}
		\newcommand{\fullref}[1]{%
			\ref{#1}\ (\small\color{hbrs_scheme5}``\protect{\nameref{#1}}''\normalsize\color{black})%
			}%
	% DEFINITION _________________________________

	\newcommand{\AddDefinition}[3] {
		\begin{addmargin}[0.5cm]{0.5cm}
			\begin{quoteBox}
				\protect\bfseries{#1}\normalfont\\\textit{``#2''{} #3}
			\end{quoteBox}
		\end{addmargin}}		

	% HREF TO FILE _________________________________

		\newcommand{\LinkFile}[2] {\href{run:#1}{#2}}

	% HREF TO FILE _________________________________

		\newcommand{\textSep}[1] {\bfseries #1 --- \normalfont}

	% LISTINGS _________________________________
			
		\definecolor{mygreen}{RGB}{28,172,0} % color values Red, Green, Blue
		\definecolor{mylilas}{RGB}{170,55,241}
				
		\lstset{language=Matlab,%
		basicstyle=\small,
		    %basicstyle=\color{red},
		    breaklines=true,%
		    morekeywords={matlab2tikz},
		    keywordstyle=\color{blue},%
		    morekeywords=[2]{1}, keywordstyle=[2]{\color{black}},
		    identifierstyle=\color{black},%
		    stringstyle=\color{mylilas},
		    commentstyle=\color{mygreen},%
		    showstringspaces=false,%without this there will be a symbol in the places where there is a space
		    numbers=left,%
		    xleftmargin=2em,
		    framexleftmargin=1.5em,
		    numberstyle={\tiny \color{black}},% size of the numbers
		    numbersep=9pt, % this defines how far the numbers are from the text
		    emph=[1]{for,end,break},emphstyle=[1]\color{red}, %some words to emphasise
		    %emph=[2]{word1,word2}, emphstyle=[2]{style},    
		}
		
		\lstnewenvironment{pythonlisting}{
		% Python style for highlighting
		\lstset{
		language=Python,
		basicstyle=\small,
		morekeywords={self},              % Add keywords here
		keywordstyle=\color{drawio_blue},
		emph={MyClass,__init__},          % Custom highlighting
		emphstyle=\color{drawio_red},    % Custom highlighting style
		stringstyle=\color{drawio_green},
		frame=tb,                         % Any extra options here
		showstringspaces=false
		}}{}
		
		\lstnewenvironment{pythonfunction}{
		\lstset{
		language=Python,
		breaklines=true,
		basicstyle=\small,
		tabsize=2,
		numbers=none,
		numbersep = 0pt,
		 emph=[1]{cv},emphstyle=[1]\color{hbrs_scheme1},
		 emph=[2]{calibrateCamera, getOptimalNewCameraMatrix, undistort, findChessboardCorners, Rodrigues}, emphstyle=[2]\bfseries\color{hbrs_scheme4},
		 emph=[3]{objP, imgP, imgSize, patternSize}, emphstyle=[3]\color{drawio_orange},
		 emph=[4]{mtx, dist, newmtx, dst, corners}, emphstyle=[4]\color{drawio_green},
		 escapechar=§
		}}{}
		
%\newcommand{\printpythonfunction}[1] {
%\begin{tcolorbox}[colback=green!5!white,colframe=red!75!black]
%\begin{pythonfunction}[style=tcblatex]
%#1%
%\end{pythonfunction}
%\end{tcolorbox}
%}
\newcommand{\pythonfunctionBegin} {
\begin{tcolorbox}[enhanced, colback=hbrs_scheme1!1!white,colframe=hbrs_scheme5!75!black, drop fuzzy shadow southeast]
\begin{pythonfunction}[style=tcblatex]
}
\newcommand{\pythonfunctionEnd} {
\end{pythonfunction}
\end{tcolorbox}
}

\newcommand{\printpythonfunction}[1] {%
%\begin{tcolorbox}[colback=green!5!white,colframe=red!75!black]%
%\begin{singlespace}\begin{lstlisting}[style=tcblatex]%
%#1%
%\end{lstlisting}\end{singlespace}%
%\end{tcolorbox}%
\begin{pythonfunction}%
#1%
\end{pythonfunction}%
}
		
	% TODO _________________________________

	\newtcolorbox{todoBox}[1]{colback=todo_scheme!5!white,colframe=todo_scheme!75!white, leftrule=3mm,drop shadow southeast, title=ToDo,fonttitle=\bfseries}
	\newcommand{\ToDo}[1] {
			\begin{addmargin}[0.5cm]{0.5cm}
				\begin{todoBox}
					\textit{#1}
				\end{todoBox}
			\end{addmargin}}
	
	% DELIVERABLE _________________________________

	\newtcolorbox{deliverableBox}[1]{enhanced, 
	colback=drawio_green!5!white,
	colframe=drawio_green!75!white, 
	leftrule=15mm,
	drop shadow southeast, 
	title=test, 
	detach title,
	fonttitle=\bfseries,
	overlay={
        \node[minimum width=1cm, anchor=south,yshift=-0.45cm, xshift=0.8cm] at (frame.west) {\LARGE\bfseries{\faFlag}};
    }}
	\newcommand{\Deliverable}[1] {
			\begin{addmargin}[0.5cm]{2.5cm}
				\begin{deliverableBox}
					{#1}{#1}
				\end{deliverableBox}
			\end{addmargin}}
			
			
	% RISK _________________________________


	\newtcolorbox{riskBox}[1]{enhanced, 
	colback=drawio_red!5!white,
	colframe=drawio_red!75!white, 
	leftrule=15mm,
	drop shadow southeast, 
	title=test, 
	detach title,
	fonttitle=\bfseries,
	overlay={
        \node[minimum width=1cm, anchor=south,yshift=-0.5cm, xshift=0.8cm] at (frame.west) {\LARGE\bfseries{\faBolt}};
    }}
	\newcommand{\Risk}[1] {
			\begin{addmargin}[0.5cm]{2.5cm}
				\begin{riskBox}
					{#1}{#1}
				\end{riskBox}
			\end{addmargin}}
			
	% RESUMEE _________________________________

	\newtcolorbox{resumeeBox}[1]{enhanced, 
	colback=hbrs_scheme6!10!white,
	colframe=hbrs_scheme6!55!white, 
	leftrule=3mm,
	toprule=1mm,
	bottomrule=1mm,
	rightrule=1mm,
	drop shadow southeast, 
	title=Resumee,fonttitle=\bfseries, 
	attach boxed title to top left={yshift=-8mm,yshifttext=-1mm}, 
	boxed title style={size=small,opacityback=0, opacityframe=0},
%	underlay boxed title={
%\path [draw=hbrs_scheme6!55!white, line width=0.5mm, fill=hbrs_scheme4!25!white]
%([yshift=-0.7mm, xshift=-0.5mm] title.north east) 
%--([yshift=-0.7mm, xshift=6mm] title.north east) 
%--([xshift=3mm] title.east)
%--([yshift=0.7mm,  xshift=6mm] title.south east)
%--([yshift=0.7mm,  xshift=-0.5mm] title.south east) ; }}
underlay boxed title={
\path [draw=hbrs_scheme6!65!white, line width=0.5mm, fill=hbrs_scheme4!25!white]
([ xshift=-0.5mm] title.north west) 
--([xshift=3mm] title.north east) 
--([] title.east)
--([xshift=3mm] title.south east)
--([xshift=-0.5mm] title.south west) ; }}

	\newcommand{\Resumee}[1] {
			\begin{addmargin}[0.5cm]{0.5cm}
				\begin{resumeeBox}
					\protect\vspace{1cm}
					\textit{#1}
				\end{resumeeBox}
			\end{addmargin}}
			
	% LAPLACE _________________________________
			
	\newcommand{\vlaplace}[1][]{\mbox{\setlength{\unitlength}{0.1em}%
                            \begin{picture}(10,20)%
                              \put(3,2){\circle{4}}%
                              \put(3,4){\line(0,1){12}}%
                              \put(3,18){\circle*{4}}%
                              \put(10,7){#1}
                            \end{picture}%
                           }%
                     }%

\newcommand{\vLaplace}[1][]{\mbox{\setlength{\unitlength}{0.1em}%
                            \begin{picture}(10,20)%
                              \put(3,2){\circle*{4}}%
                              \put(3,4){\line(0,1){12}}%
                              \put(3,18){\circle{4}}%
                              \put(10,7){#1}
                            \end{picture}%
                           }%
                     }%  

% #####################################
% THESIS TEMPLATE (HBRS BACHELOR & MASTER)
% Author: Ilka Salomon
% E-Mail: ilka.salomon@gmx.de
% #####################################